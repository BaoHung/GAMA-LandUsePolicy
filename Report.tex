\documentclass{article}
\usepackage{listings}
\usepackage{graphicx}
\usepackage{epstopdf}
\usepackage[english]{babel}

\title{%
  Models to analyse land use policies in the Mekong Delta \\
  \large Subject 1: Improving the realism of the individual economic choices}

\author{NGUYEN Vu Bao Hung}
\date{June 16, 2018}
\lstset{breaklines=true} 

\begin{document}
\maketitle

\section{Introduction}
I chose the topic 1 which is to add the consideration of investment, cost and delay of each culture to the farmers' decisions. I then measure the rate of parcel that successfully reach the plan of 2010 to see if the new elements help steering farmers into the right track.

\section{The model}
I use cost.csv for the cost of change.\\
I created investment.csv to store the investment required in the change of culture.\\
I added a delay column in the lut.csv file to store the delay required of each culture.\\
After that, in the model, I added weight parameter, criteria element, and calculation method for each of the 3 new elements.

\section{The Experiment Plan and Exploration}
With the exhaustive method, the experiment has 3 parameters which are the weights of the new elements (cost, investment and delay)\\
They range from 0.2 to 0.8 with the step of 0.3, it results in 27 runs of simulation.\\
I measure the percentage of parcel that has "land use" the same as the 2010 plan.\\
The results are stored in a csv file for latter analysis.

\section{Data analysis}
By using R, I have computed the Sum Square of each weight and its ratio on the effect of farmers making the right decisions.\\
The result shows that the weight of delay takes more than 60\% of the effect on the result. It is the most important parameter.\\
Meanwhile, due to stochasticity, the residual is quite high 35\% which indicates that the model is not very stable.

\section{Calibration}
The calibration was implemented in an IDEA project. It calls the headless mode of GAMA and gets the result of simulation in a series of xml files.

\end{document}
